\documentclass[twocolumn]{article}
\usepackage[portuguese]{babel}
\usepackage{indentfirst}
\usepackage{svg}
\usepackage{cuted}
\usepackage{caption}

\author{Gabriel G. de Brito\\ Departamento de Informática - Universidade Federal do Paraná}
\title{Implementação Simulada de uma Unidade Lógica Aritmética}

\begin{document}

\maketitle

\section{Introdução}

Neste trabalho documento o processo da criação e o funcionamento de uma Unidade
Lógica Aritmética (ULA) simulada utilizando o simulador de circuitos
Digital\footnote{https://github.com/hneemann/Digital}.

Como foi requisitado, a ULA possui duas entradas de 16 bits para operandos e uma
entrada de 3 bits para a operação. Há também uma saída de 16 bits para o
resultado e uma saída de 1 bit que identifica se o resultado é 0. As operações
são, em ordem, adição, subtração, AND, OR, NOT, GT (maior que), LT (menor que) e
EQ (igual a). Todas as operações (com exceção da NOT, que trabalha somente com a
entrada ``A'') trabalham com ambos os operandos. Além disso, todas as operações
são feitas em complemento de 2.

\section{Circuito Principal}

O circuito principal liga as entradas em todas as operações simultaneamente. A
entrada da função é ligada em multiplexadores que selecionam o resultado da
operação desejada. É importante notar que ambas as operações de soma e subtração
dividem o mesmo somador. A função seleciona então se a segunda entrada do
somador é o complemento lógico do segundo operando ou não e, da mesma maneira,
seleciona se a entrada de carry é 1 ou 0. Dessa maneira, o somador pode ser
usado para as duas operações, transformando a operação de subtração em uma soma
com o complemento de 2 da entrada ``B''.

Não obstante, também deve-se notar o detalhe das operações de comparação. Como
seus resultados são de apenas 1 bit, porém a entrada do multiplexador e a saída
do resultado são de 16 bits, deve-se usar um extensor de sinal para que o
resultado seja propriamente propagado. Dessa maneira, o resultado é ``todos os
bits setados'' ou ``todos os bits zerados'' para ``verdadeiro'' e ``falso'',
respectivamente.

A saída ``zero'' é ligada à saída EQ de um comparador cujos operandos são a
saída da operação e uma constante 0. Dessa maneira, ela é 1 quando o resultado
da operação for 0 e 0 caso contrário.

\begin{figure}[h]
    \centering
    \includesvg[width=\columnwidth]{ula}
    \caption{Circuito principal da ULA}
\end{figure}

\section{Somador}

Um somador completo de 1 bit é um circuito simples, que recebe três entradas
(dois operandos e um sinal de carry) e possui duas saídas (o resultado e o sinal
de carry). É possível implementar um somador maior apenas ligando o carry de
saída de um somador ao carry de entrada do próximo. Dessa maneira, o somador de
16 bits é implementado utilizando-se de 16 somadores de 1 bit. Não é necessário
se preocupar com números negativos, pois a representação em complemento de 2
mantém as propriedades aritméticas de números positivos e negativos.

A saída de carry do último somador é utilizada como saída de carry do circuito.
Nesse trabalho, não há tratamento de overflow, porém é possível implementá-lo
verificando essa saída e o sinal dos números. Esse circuito também é utilizado
para a subtração, bastando inverter o segundo operando e a entrada de carry. Por
ser um trabalho simples e fora do escopo do somador em si, isso é implementado
diretamente no circuito principal.

\begin{figure}[h]
    \includesvg[width=\columnwidth]{somador}
    \caption{Somador de 1 bit}
\end{figure}

\begin{strip}
    \centering \includesvg[width=\textwidth]{somador-16bit}
    \captionof{figure}{Somador de 16 bits}
\end{strip}

\section{Comparador}

O comparador é um circuito que recebe dois operandos e possui três saídas: GT,
LT e EQ. É, com certeza, o circuito mais complexo da ULA. Para um único bit, a
saída EQ vem de um simples XNOR. Para as outras duas saídas, usa-se um AND com
uma das entradas negadas. Nessa forma, não é possível ligar vários comparadores
para criar um maior de maneira simples. Utilizando de dois comparadores de 1
bit, podemos criar um comparador de 2 bits ligando as saídas EQ dos dois em uma
porta AND, e comparando todas as combinações possíveis para criar as outras duas
saídas.

Para o comparador de 16 bits, porém, precisamos de um circuito que possa ser
ligado em cascata. Para isso, criamos um comparador especial de 2 bits
utilizando o circuito que já foi criado. Ele possui 3 entradas que representam o
resultado da operação de comparação anterior, e usa a mesma lógica que o
comparador de 2 bits normal.

\begin{strip}
    \centering \includesvg[width=\textwidth]{comparador-16bit}
    \captionof{figure}{Comparador de 16 bits}
\end{strip}

\begin{figure}
    \includesvg[width=\columnwidth]{comparador-1bit}
    \caption{Comparador de 1 bit}
\end{figure}

\begin{figure}
    \includesvg[width=\columnwidth]{comparador-2bit}
    \caption{Comparador de 2 bits}
\end{figure}

\begin{figure}
    \includesvg[width=\columnwidth]{comparador-2bit-cascata}
    \caption{Comparador de 2 bits que permite a ligação em cascata}
\end{figure}

Como os números estão em complemento de 2, precisamos fazer o tratamento do
sinal na comparação. Para isso, separamos o bit mais significativo de cada
número e fazemos uma comparação inversa (isso pois, caso os bits sejam
diferentes, e A for 1, significa, por exemplo, que A é menor que B, pois A é
negativo e B positivo). Dessa maneira, sobram 15 bits para comparar. Separamos o
segundo bit mais significativo de cada operando e comparamos utilizando um
comparador de 1 bit. Esse comparador servirá como ``bootstrap'', pois seu
resultado será ligado a um comparador de 2 bits que pode ser cascadeado. Os 14
bits restantes são separados em grupos de dois, e comparados normalmente.

Ainda que seja viável implementar o comparador de 16 bits utilizando
comparadores de 4 bits, foi optado por utilizar comparadores de 2 bits pois após
comparar o sinal e o segundo bit, sobram 14 bits para comparar, que podem ser
mais facilmente divididos em grupos de 2. Separar em grupos de 4 ainda exigiria
um circuito para comparar somente dois bits, aumentando a quantidade de módulos
do sistema.

\section{Considerações Finais}

O trabalho foi concluído com sucesso, e todos os requisitos foram atendidos.
Testes demonstram que o circuito funciona corretamente. É possível implementar
tal circuito de maneira mais barata, porém para o escopo do trabalho acredito
que a solução apresentada é a mais simples.

\end{document}
